\documentclass[a4paper]{article}

\usepackage[utf8]{inputenc}
\usepackage[portuguese]{babel}
\usepackage{a4wide}
\usepackage[pdftex]{hyperref}
\usepackage{graphicx}
\usepackage{wrapfig}
\usepackage{amsmath}
\usepackage{caption}
\usepackage{subcaption}
\usepackage{tikz}
\usepackage{pgfplots}
\pgfplotsset{compat=1.10}
\usepgfplotslibrary{fillbetween}
\usepackage{pgfplots}
\usetikzlibrary{patterns}



\begin{document}

\begin{titlepage}
\begin{center}


\includegraphics[width=0.4\textwidth]{logo}\\[0.3cm]

{\large Universidade do Minho - Escola de Engenharia}\\[0.5cm]

{\large Relatório do trabalho prático de Desenvolvimento de Sistemas de Software}\\[0.5cm]

\begin{abstract}

Neste relatório será feita uma abordagem inicial ao projeto de Desenvolvimentos de Sistemas de Software ao qual está associado o desenvolvimento de um programa, em Java, responsável pela gestão dos turnos de um curso. Assim, este documento apresenta detalhadamente a perspetiva tomada pelo grupo em relação ao problema proposto pela equipa docente de DSS.

\hspace{3mm}
\end{abstract}

% Title
\rule{\linewidth}{0.5mm} \\[0.4cm]
{ \huge \bfseries Sistema de Gestão de Turnos Práticos \\[0.4cm] }
\rule{\linewidth}{0.5mm} \\[1.5cm]

% Author and supervisor
\noindent
\begin{minipage}{0.4\textwidth}
  \begin{flushleft} \large
    \emph{Autores :}\\
    Diana Costa \textsc{(A78985)}\\
    \includegraphics[width=1.5cm]{diana}\break
    Marcos Pereira \textsc{(A79116)}\\
    \includegraphics[width=1.5cm]{marcos}\break
    Sérgio Oliveira\textsc{(A77730)}\\
    \includegraphics[width=1.5cm]{sergio}\break
    Vitor Castro\textsc{(A77870)}\\
    \includegraphics[width=1.5cm]{vitor}\break
  \end{flushleft}
\end{minipage}%
\vfill

% Bottom of the page
{\large Versão 1.0 \\ \today}

\end{center}
\end{titlepage}


\pagebreak
\tableofcontents

\pagebreak
\section{Introdução}
\label{sec:1}

Este projeto tem como objetivo desenvolver um sistema capaz de alocar e gerir os turnos de um curso. A sua execução permitirá consolidar conhecimentos ao nível da programação em linguagens de objetos e introduz abordagens organizadas e estruturadas de desenvolvimento de software a partir de modelação e representação de dados em UML 2.x. 
Nesta fase intermédia do trabalho foi-nos então sugerido o concebimento da representação gráfica ou diagramas nessa linguagem.

\section{Problema}
\label{sec:2}

\hspace{3mm}

Pretende-se desenvolver um sistema que atribui turnos ou um horário aos alunos do curso e que possibilita a ocorrência de trocas entre estes. As trocas estão condicionadas pela existência de dois alunos interessados em trocar de turno, no caso de pertencerem ao regime normal e, são feitas sem condicionalismos no caso do aluno ser trabalhador estudante.
Neste momento, deve apresentar-se uma análise de requisitos da qual resultará um Modelo de Domínio, um Modelo de Use Case e uma proposta de interface gráfica.

\section{Solução}
\label{sec:3}


A nossa solução baseia-se na UML - Unified Modeling Language, linguagem útil na elaboração, modelação e documentação da estrutura de projetos de software e de sistemas orientados a objetos. Portanto, auxilia os developers de programas a visualizarem os seus sistemas através de diagramas padronizados.
Até ao momento, a nossa solução foi implementada com base em:

\begin{itemize}
    \item Esquema de Domínio;
    \item Esquema de Use Case;
    \item Especificação dos Use Case;
    \item Proposta para a Interface Gráfica
\end{itemize}

\subsection{Esquema de Domínio}
O modelo de domínio analisa o problema de uma perspetiva concetual e é a representação gráfica das classes do programa e dos seus atributos assim como o relacionamento entre estas.

\includegraphics[width=12cm]{dominio}\break

\subsection{Esquema de Use Case}

\includegraphics[width=12cm]{diagramausecase}\break
Diagrama que mostra o que o programa faz do ponto de vista do utilizador ou seja, consegue mostrar as principais funcionalidades do sistema e as suas interações com os atores do sistema. Por isso, os diagramas de use case são compostos por atores (utilizadores do sistema), por use case (funcionalidade) e pelas comunicações entre atores e use case.

\subsection{Especificação textual dos Use Case}
Dado que um cenário é uma sequência de passos da interação entre ator e sistema, então a especificação textual dos casos de uso é um documento que descreve os vários cenários possíveis entre as comunicações de um mesmo objetivo ou funcionalidade.

\subsubsection{Atribuir docente a turno}
\includegraphics[width=12cm]{atribuidocenteturno}\break

\subsubsection{Atribuir turno a aluno}
\includegraphics[width=12cm]{atribuiturnoaluno}\break

\subsubsection{Atribuir UC a turno}
\includegraphics[width=12cm]{atribuiucaturno}\break

\subsubsection{Registar docente}
\includegraphics[width=12cm]{registardocente}\break

\subsubsection{Criar turnos}
\includegraphics[width=12cm]{criarturnos}\break

\subsubsection{Criar UC}
\includegraphics[width=12cm]{criaruc}\break

\subsubsection{Efectuar Login}
\includegraphics[width=12cm]{efectuarlogin}\break

\subsubsection{Remover aluno de turno}
\includegraphics[width=12cm]{removeralunoturno}\break

\subsubsection{Definir número máximo de alunos por turno prático}
\includegraphics[width=12cm]{definiralunos}\break

\subsubsection{Inscreve aluno em turno}
\includegraphics[width=12cm]{inscreveralunoturno}\break

\subsubsection{Efectuar Registo}
\includegraphics[width=12cm]{efectuarregisto}\break

\subsubsection{Propôr Troca}
\includegraphics[width=12cm]{proportroca}\break



\subsection{Interface Gráfica}

\subsubsection{Login/Registo}
\includegraphics[width=12cm]{interface_1}\break

\subsubsection{Minha Área}
\includegraphics[width=12cm]{interface_2}\break

\subsubsection{Minhas Trocas}
\includegraphics[width=12cm]{interface_3}\break

\subsubsection{Ver Lista de Trocas Pendentes}
\includegraphics[width=12cm]{interface_4}\break



\pagebreak
\section{Conclusões}
\label{sec:4}

\hspace{3mm}
\end{document}
